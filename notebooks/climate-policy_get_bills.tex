
\begin{pyin}
import congress
import datetime
import dotenv
import json
import time
from urllib.parse import urlencode
\end{pyin}

\begin{pyin}
api_key = dotenv.get_key('../.env', 'CONGRESS_API_KEY')
\end{pyin}

\begin{pyin}
c = congress.Congress(api_key)
\end{pyin}

\begin{pyin}
def convert_to_bill_id(bill_number):
    bill_id = bill_number.lower()
    bill_id = bill_id.replace(' ', '')
    bill_id = bill_id.replace('.', '')
    return bill_id
\end{pyin}

\begin{pyin}
def get_related_subjects(keyword_filename):
    # create list of subject queries from ../models/story_keywords.txt
    with open(keyword_filename, 'r') as f:
        keywords = f.read().splitlines()

    # get subjects related to climate
    subjects = set()

    for keyword in keywords:
        query = urlencode({'query': keyword.lower()})
        kw_subjects = c.fetch('bills/subjects/search.json?' + query)['subjects']
        for subj in kw_subjects:
            if subj['has_bills'] == 'true':
                subjects.add(subj['url_name'])

    return subjects
\end{pyin}

\begin{pyin}
def bills_for_subjects(subjects, logging=False):
    bills = {}
    for subject in subjects:
        if logging:
            print('Getting bills for subject: ' + subject + '...', end=' ')
        fetched_bills = c.fetch('bills/subjects/' + subject + '.json', parse=lambda r: r['results'])
        if logging:
            print('%d bills found.' % len(fetched_bills))
        for b in fetched_bills:
            bills[b['bill_id']] = b
    return bills
\end{pyin}

\begin{pyin}
subjects = get_related_subjects(keyword_filename='../models/bill_keywords.txt')
\end{pyin}

\begin{pyin}
\## Save bill subjects
with open('../data/bill_subjects.txt', 'w') as f:
    for subj in subjects:
        f.write(subj + '\n')
\end{pyin}

\begin{pyin}
bills = bills_for_subjects(subjects)
\end{pyin}

\begin{pyin}
\## Set bill last action cutoff date
# 27 months is the maximum reachback we can do with the AYLIEN API
now = datetime.datetime.now()
cutoff_month = now.month - 3
cutoff_day = now.day
cutoff_year = now.year - 2

if cutoff_month <= 0:
    cutoff_month += 12
    cutoff_year -= 1

cutoff_date = datetime.datetime(cutoff_year, cutoff_month, cutoff_day)
\end{pyin}

\begin{pyin}
\## Remove all bills whose last action was earlier than the cutoff
recent_bills = {}
for bill_id, bill in bills.items():
    last_action_date = datetime.datetime.strptime(bill['latest_major_action_date'], '%Y-%m-%d')
    if last_action_date >= cutoff_date:
        recent_bills[bill_id] = bill
\end{pyin}

\begin{pyin}
\## Save bill ids
with open('../data/bill_ids.txt', 'w') as f:
    for bill_id in recent_bills.keys():
        f.write(bill_id + '\n')
\end{pyin}

\begin{pyin}
def get_latest_vote(bill):
    latest_house_roll_call = 0
    house_api_url = None
    latest_senate_roll_call = 0
    senate_api_url = None

    for vote in bill['votes']:
        if vote['chamber'] == 'House':
            if int(vote['roll_call']) > latest_house_roll_call:
                latest_house_roll_call = int(vote['roll_call'])
                house_api_url = vote['api_url']
        elif vote['chamber'] == 'Senate':
            if int(vote['roll_call']) > latest_senate_roll_call:
                latest_senate_roll_call = int(vote['roll_call'])
                senate_api_url = vote['api_url']

    house = None
    senate = None

    if house_api_url:
        house_api_uri = house_api_url.split('https://api.propublica.org/congress/v1/')[1]
        house = c.fetch(house_api_uri, parse=lambda r: r['results']['votes']['vote'])

    if senate_api_url:
        senate_api_uri = senate_api_url.split('https://api.propublica.org/congress/v1/')[1]
        senate = c.fetch(senate_api_uri, parse=lambda r: r['results']['votes']['vote'])

    return house, senate
\end{pyin}

\begin{pyin}
\## Restructure for the data pipeline
# Our schema is the following:
# {
#   <bill_id>: {
#     'bill_id': <bill_id> (str),
#     'congress': <congress> (int)
#     'bill_number': <bill_number> (str),
#     'short_title': <short_title> (str),
#     'introduced': <introduced_date> (str),
#     'last_action': <last_action> (str),
#     'last_action_date': <last_action_date> (str),
#     'house_vote': {
#       'date': <date> (str),
#       'passed': <result> (bool),
#       'total_proportion': <proportion_of_total> (float),
#       'democrats': <proportion_of_democrats> (float),
#       'republicans': <proportion_of_republicans> (float),
#       'independents': <proportion_of_independents> (float),
#     },
#     'senate_vote': {
#       'date': <date> (str),
#       'passed': <result> (bool),
#       'total_proportion': <proportion_of_total> (float),
#       'democrats': <proportion_of_democrats> (float),
#       'republicans': <proportion_of_republicans> (float),
#       'independents': <proportion_of_independents> (float)
#     }
#   }
# }
# Note: the schema uses strings for dates since datetime objects are not JSON serializable
def get_bill_restruct(bill_id, bill):
    bill_r = {
        'bill_id': bill_id.split('-')[0],
        'congress': int(bill_id.split('-')[1]),
        'bill_number': bill['number'],
        'short_title': bill['short_title'],
        # 'introduced': datetime.datetime.strptime(bill['introduced_date'], '%Y-%m-%d'),
        'introduced': bill['introduced_date'],
        'last_action': bill['latest_major_action'],
        # 'last_action_date': datetime.datetime.strptime(bill['latest_major_action_date'], '%Y-%m-%d'),
        'last_action_date': bill['latest_major_action_date'],
        'house_vote': {},
        'senate_vote': {}
    }

    if not bill['votes']:
        return bill_r

    house_vote, senate_vote = get_latest_vote(bill)

    if house_vote:
        house_passed = house_vote['result'] == 'Passed'

        house_dems_yes = house_vote['democratic']['yes']
        house_dems_no = house_vote['democratic']['no']
        all_house_dems = house_dems_yes + house_dems_no

        house_reps_yes = house_vote['republican']['yes']
        house_reps_no = house_vote['republican']['no']
        all_house_reps = house_reps_yes + house_reps_no

        house_ind_yes = house_vote['independent']['yes']
        house_ind_no = house_vote['independent']['no']
        all_house_ind = house_ind_yes + house_ind_no

        bill_r['house_vote'] = {
            # 'date': datetime.datetime.strptime(house_vote['date'], '%Y-%m-%d'),
            'date': house_vote['date'],
            'passed': house_passed,
            'total_proportion': ((house_dems_yes + house_reps_yes + house_ind_yes) / (all_house_dems + all_house_reps + all_house_ind)) if (all_house_dems + all_house_reps + all_house_ind) > 0 else None,
            'democrats': (house_dems_yes / all_house_dems) if all_house_dems > 0 else None,
            'republicans': (house_reps_yes / all_house_reps) if all_house_reps > 0 else None,
            'independents': (house_ind_yes / all_house_ind) if all_house_ind > 0 else None,
        }

    if senate_vote:
        senate_passed = senate_vote['result'] == 'Passed'

        senate_dems_yes = senate_vote['democratic']['yes']
        senate_dems_no = senate_vote['democratic']['no']
        all_senate_dems = senate_dems_yes + senate_dems_no

        senate_reps_yes = senate_vote['republican']['yes']
        senate_reps_no = senate_vote['republican']['no']
        all_senate_reps = senate_reps_yes + senate_reps_no

        senate_ind_yes = senate_vote['independent']['yes']
        senate_ind_no = senate_vote['independent']['no']
        all_senate_ind = senate_ind_yes + senate_ind_no

        bill_r['senate_vote'] = {
            # 'date': datetime.datetime.strptime(senate_vote['date'], '%Y-%m-%d'),
            'date': senate_vote['date'],
            'passed': senate_passed,
            'total_proportion': ((senate_dems_yes + senate_reps_yes + senate_ind_yes) / (all_senate_dems + all_senate_reps + all_senate_ind)) if (all_senate_dems + all_senate_reps + all_senate_ind) > 0 else None,
            'democrats': (senate_dems_yes / all_senate_dems) if all_senate_dems > 0 else None,
            'republicans': (senate_reps_yes / all_senate_reps) if all_senate_reps > 0 else None,
            'independents': (senate_ind_yes / all_senate_ind) if all_senate_ind > 0 else None
        }

    return bill_r
\end{pyin}

\begin{pyin}
bills_restruct = {}

# this is to resume in case an issue occurs
retry_count = 0
MAX_RETRIES = 5

while retry_count < MAX_RETRIES:
    try:
        for bill_id, _ in recent_bills.items():
            if bill_id in bills_restruct:
                continue
            bill = c.bills.get(*bill_id.split('-'))
            bills_restruct[bill_id] = get_bill_restruct(bill_id, bill)
        break
    except Exception as e:
        print(e)
        retry_count += 1
        print(f'Error occurred, retry {retry_count} of {MAX_RETRIES}...')
        time.sleep(5)
\end{pyin}

\begin{pyin}
\## write entire bill data to file
with open('../data/bill_data.json', 'w') as f:
    json.dump(bills_restruct, f, indent=2)
\end{pyin}
